\documentclass[12pt]{article}

\usepackage{bm}
\usepackage{amsmath, mathtools}
\usepackage{amsfonts}
\usepackage{amssymb}
\usepackage{graphicx}
\usepackage{colortbl}
\usepackage{xr}
\usepackage{hyperref}
\usepackage{longtable}
\usepackage{xfrac}
\usepackage{tabularx}
\usepackage{float}
\usepackage{siunitx}
\usepackage{booktabs}
\usepackage{multirow}
\usepackage[section]{placeins}
\usepackage{caption}


%\usepackage{refcheck}

\hypersetup{
    bookmarks=true,         % show bookmarks bar?
      colorlinks=true,       % false: boxed links; true: colored links
    linkcolor=red,          % color of internal links (change box color with linkbordercolor)
    citecolor=green,        % color of links to bibliography
    filecolor=magenta,      % color of file links
    urlcolor=cyan           % color of external links
}

%% Comments
\newif\ifcomments\commentstrue

\ifcomments
\newcommand{\authornote}[3]{\textcolor{#1}{[#3 ---#2]}}
\newcommand{\todo}[1]{\textcolor{red}{[TODO: #1]}}
\else
\newcommand{\authornote}[3]{}
\newcommand{\todo}[1]{}
\fi

\newcommand{\wss}[1]{\authornote{blue}{SS}{#1}}
\newcommand{\bmac}[1]{\authornote{red}{BM}{#1}}

\newcommand{\colZwidth}{1.0\textwidth}
\newcommand{\blt}{- } %used for bullets in a list
\newcommand{\colAwidth}{0.13\textwidth}
\newcommand{\colBwidth}{0.82\textwidth}
\newcommand{\colCwidth}{0.1\textwidth}
\newcommand{\colDwidth}{0.05\textwidth}
\newcommand{\colEwidth}{0.8\textwidth}
\newcommand{\colFwidth}{0.17\textwidth}
\newcommand{\colGwidth}{0.5\textwidth}
\newcommand{\colHwidth}{0.28\textwidth}
\newcounter{defnum} %Definition Number
\newcommand{\dthedefnum}{GD\thedefnum}
\newcommand{\dref}[1]{GD\ref{#1}}
\newcounter{datadefnum} %Datadefinition Number
\newcommand{\ddthedatadefnum}{DD\thedatadefnum}
\newcommand{\ddref}[1]{DD\ref{#1}}
\newcounter{theorynum} %Theory Number
\newcommand{\tthetheorynum}{T\thetheorynum}
\newcommand{\tref}[1]{T\ref{#1}}
\newcounter{tablenum} %Table Number
\newcommand{\tbthetablenum}{T\thetablenum}
\newcommand{\tbref}[1]{TB\ref{#1}}
\newcounter{assumpnum} %Assumption Number
\newcommand{\atheassumpnum}{P\theassumpnum}
\newcommand{\aref}[1]{A\ref{#1}}
\newcounter{goalnum} %Goal Number
\newcommand{\gthegoalnum}{P\thegoalnum}
\newcommand{\gsref}[1]{GS\ref{#1}}
\newcounter{instnum} %Instance Number
\newcommand{\itheinstnum}{IM\theinstnum}
\newcommand{\iref}[1]{IM\ref{#1}}
\newcounter{reqnum} %Requirement Number
\newcommand{\rthereqnum}{P\thereqnum}
\newcommand{\rref}[1]{R\ref{#1}}
\newcounter{lcnum} %Likely change number
\newcommand{\lthelcnum}{LC\thelcnum}
\newcommand{\lcref}[1]{LC\ref{#1}}

\newcommand{\tclad}{T_\text{CL}}
\newcommand{\degree}{\ensuremath{^\circ}}
\newcommand{\progname}{SWHS}

\newcounter{mnum}
\newcommand{\mthemnum}{M\themnum}
\newcommand{\mref}[1]{M\ref{#1}}


%\oddsidemargin 0mm
%\evensidemargin 0mm
%\textwidth 160mm
%\textheight 200mm
\usepackage{fullpage}


\begin{document}

\title{Module Interface Specification for Solar Water Heating Systems
  Incorporating Phase Change Material} 
\author{Maya Grab and Brooks MacLachlan}
\date{\today}
	
\maketitle

\tableofcontents

\section{Introduction}
The following document details the Module Interface Specifications for the implemented 
modules in a program simulation Solar Water Heating System with Phase Change Material.
 It is intended to ease navigation through the program for design and maintenance purposes.\\
  Complementary documents include the System Requirement Specifications and Module Guide. 

\section{Module Decomposition}
The following table is taken directly from the Module Guide document for this project.
\begin{table}[!h]
\centering
\begin{tabular}{p{0.3\textwidth} p{0.6\textwidth}}
\toprule
\textbf{Level 1} & \textbf{Level 2}\\
\midrule

{Hardware-Hiding Module} & ~ \\
\midrule

\multirow{7}{0.3\textwidth}{Behaviour-Hiding Module} & Input Format Module\\
& Input Parameters Module\\
& Input Verification Module\\
& Output Format Module\\
& Output Verification Module\\
& Temperature ODEs Module\\
& Energy Equations Module\\ 
& Control Module\\
\midrule

\multirow{3}{0.3\textwidth}{Software Decision Module} & {Sequence Data Structure Module}\\
& ODE Solver Module\\
& Plotting Module\\
\bottomrule

\end{tabular}
\caption{Module Hierarchy}
\label{TblMH}
\end{table}
%%%%%%%%%%%%%%%%%%%%%%%%%%%%%%%%%%%%%%%%%%%%%%%%%%%%%%%

\section{MIS of Control Module}

\subsection{Module Name: main.m}

\subsection{Uses}

%\subsubsection{Imported Constants}


\subsubsection{Imported Data Types}
\textbf{Uses} Input Parameters Module \textbf{Imports} params := structure 

\subsubsection{Imported Access Programs}
\textbf{Uses} Input Format Module \textbf{Imports} load\_{params} \\
\textbf{Uses} Energy Module \textbf{Imports} energy1, energy2, energy3 \\
\textbf{Uses} Temperature Module \textbf{Imports} 
temperature1, temperature2, temperature3 \\
\textbf{Uses} Event Module \textbf{Imports} event1, event2 \\
\textbf{Uses} Plot Module \textbf{Imports} plots \\


\subsection{Interface Syntax}

%\subsubsection{Exported Constants}
%None

%\subsubsection{Exported Data Types}
%None

\subsubsection{Exported Access Programs}
\begin{center}
\begin{tabular}{l c c c}
\hline \\
\textbf{Name} & \textbf{In} & \textbf{Out} & \textbf{Exceptions} \\ \hline
main & String (file) & \shortstack{\\ Modifies the screen environment\\
Modifies the output file to contain its state variables}  & Various  \\
\hline
\end{tabular}
\end{center}

\subsection{Interface Semantics}

\subsubsection{State Variables}
t : column vector\\
T : column vector\\
Ew : column vector\\
Ep : column vector\\
Etot : column vector

%Temperature isn't a column vector... it has two columns.

\subsubsection{Assumption}
%Assumptions from SRS?

\bmac{Should this be commented out if there is no content?}

\subsubsection{Invariant}
None

\subsubsection{Access Program Semantics}

\noindent \textbf{Input:}\\
Main.m will accept a valid file name string which is accessible to the current Matlab path.
The data in the provided file must comply with the format specifications required
by the Input Format Module.\\

\noindent \textbf{Exceptions:}\\
Potential exceptions are invalid file names or paths. Other exceptions are possible within the Input Format Module and Input Verification Module due to inappropriate input.\\

\noindent \textbf{Output:}\\
Main.m will print to the screen messages regarding phase changes of the PCM.\\
Main.m will request the Output Format Module to produce a file with the  same file
name as the input string, with the extension .out. The file will contain the 
state variables of main at the end of the simulation.\\
Main.m will request the Plotting Module to produce energy and temperature graphs
from its state variables.

%%%%%%%%%%%%%%%%%%%%%%%%%%%%%%%%%%%%%%%%%%%%%

\section{MIS of Input Parameters Module}

\subsection{Module Name: ParamT} % ??

%\subsection{Uses}

%\subsubsection{Imported Constants}

%\subsubsection{Imported Data Types}

%\subsubsection{Imported Access Programs}

\subsection{Interface Syntax}

%\subsubsection{Exported Constants}

\subsubsection{Exported Data Types}

params := structure


%\subsubsection{Exported Access Programs}

\subsection{Interface Semantics}

Params is a data structure designed to store the input information entered by the Input Format module.


%%%%%%%%%%%%%%%%%%%%%%%%%%%%%%%%%%%%%%%%%%%%%

\section{MIS of Input Format Module}

\subsection{Module Name: load\_{params}}

%\subsection{Uses}
%None

%\subsubsection{Imported Constants}
%None

\subsubsection{Imported Data Types}
\textbf{Uses} Input Parameters Module \textbf{Imports} params := structure 

%\subsubsection{Imported Access Programs}
%None

\subsection{Interface Syntax}

%\subsubsection{Exported Constants}
%None

%\subsubsection{Exported Data Types}
%None

\subsubsection{Exported Access Programs}

\begin{center}
\begin{tabular}{l c c c}
\hline
\textbf{Name} & \textbf{In} & \textbf{Out} & \textbf{Exceptions} \\ \hline
load\_{params} & String (file) & Struct & Inappropriately formatted input file \\
\hline
\end{tabular}
\end{center}


\subsection{Interface Semantics}

%\subsubsection{State Variables}
%None

%\subsubsection{Assumption}
%None

%\subsubsection{Invariant}
%None

\subsubsection{Access Program Semantics}

\noindent \textbf{Input:}\\
load\_{params} will accept a valid file name string which is accessible to the current Matlab path.
The readable data in the given file must be numeric only, with all comments 
appropriately signaled.\\

\noindent \textbf{Exceptions:}\\
Potential exceptions may occur when the given file containing input data is inappropriately formatted. \\  

\noindent \textbf{Output:}\\
When given valid data load\_{params} will return a structure containing all
inputted data under appropriate field names, in addition to some relevant 
calculated data fields.

%%%%%%%%%%%%%%%%%%%%%%%%%%%%%%%%%%%%%%%%%

\section{MIS of Input Verification Module}

\subsection{Module Name: verify\_params}

\subsubsection{Imported Data Types}
\textbf{Uses} Input Parameters Module \textbf{Imports} params := structure

\subsection{Interface Syntax}

\subsubsection{Exported Access Programs}

\begin{center}
	\begin{tabular}{l c c c}
		\hline
		\textbf{Name} & \textbf{In} & \textbf{Out} & \textbf{Exceptions} \\ \hline
		verify\_{params} & Struct & none & Various (see appendix) \\
		\hline
	\end{tabular}
\end{center}

\subsection{Interface Semantics}

\subsubsection{Access Program Semantics}

\noindent \textbf{Input:}\\
verify\_{params} will accept a data structure containing the fields created by the Input Format Module.\\

\noindent \textbf{Exceptions:}\\
Data which does not comply with the data constraints specifications detailed in
the SRS document for this project will yield one of the potential exceptions or
warnings as listed in the appendix of this document.\\  

\noindent \textbf{Output:}\\
None

%%%%%%%%%%%%%%%%%%%%%%%%%%%%%%%%%%%%%%%%%

\section{MIS of Temperature Modules}

\subsection{Module Name: temperature1; temperature2;, temperature3}

%\subsection{Uses}
%None

%\subsubsection{Imported Constants}
%None

%\subsubsection{Imported Data Types}
%None

%\subsubsection{Imported Access Programs}
%None

\subsection{Interface Syntax}

%\subsubsection{Exported Constants}
%None

%\subsubsection{Exported Data Types}
%None

\subsubsection{Exported Access Programs}

\begin{center}
\begin{tabular}{l c c c}
\hline
\textbf{Name} & \textbf{In} & \textbf{Out} & \textbf{Exceptions} \\ \hline
temperature1 & \shortstack{\\ real \\ vector \\ struct} & vector & \shortstack{\\  Vector dimensions \\ Data structure missing fields} \\ \hline
temperature2 & \shortstack{\\ real \\ vector \\ struct} & vector & \shortstack{\\ Vector dimensions \\ Data structure missing fields} \\ \hline
temperature3 & \shortstack{\\ real \\ vector \\ struct} & vector & \shortstack{\\ Vector dimensions \\ Data structure missing fields} \\ \hline
\end{tabular}
\end{center}

\subsection{Interface Semantics}

%\subsubsection{State Variables}
%None

\subsubsection{Assumption}
The Control Module and the Event Module handled the different cases correctly,
 ensuring that each temperature function necessarily receives valid numerical input 
  for the case it is meant to handle.

%\subsubsection{Invariant}
%None

\subsubsection{Access Program Semantics}

\noindent \textbf{Input:}\\
The Temperature Module functions require a time value, a vector containing at 
least two values, and a data structure containing the fields created by the 
Input Format Module. \\

\noindent \textbf{Exceptions:}\\
Potential exceptions may occur when given inappropriately sized
vectors or a data structure lacking the required fields.\\

\noindent \textbf{Output:}\\
The Temperature Module will return a vector containing numerical solutions to
the differential equations it governs. \\

\subsubsection{Considerations}
Note: the three exported access functions are implemented in three separate function files. 

%%%%%%%%%%%%%%%%%%%%%%%%%%%%%%%%%%%%%%%%

\section{MIS of Energy Modules}

\subsection{Module Name: energy1, energy2, energy3}

%\subsection{Uses}

%\subsubsection{Imported Constants}
%None

%\subsubsection{Imported Data Types}
%None

%\subsubsection{Imported Access Programs}
%None

\subsection{Interface Syntax}

%\subsubsection{Exported Constants}
%None

%\subsubsection{Exported Data Types}
%None

\subsubsection{Exported Access Programs}
\begin{center}
\begin{tabular}{l c c c}
\hline
\textbf{Name} & \textbf{In} & \textbf{Out} & \textbf{Exceptions} \\ \hline
energy1 & \shortstack{\\ matrix \\ struct} & vector & \shortstack{\\ Matrix dimensions\\ Data structure missing fields} \\ \hline
energy2 & \shortstack{\\ matrix \\ struct} & vector & \shortstack{\\ Matrix dimensions\\ Data structure missing fields} \\ \hline
energy3 & \shortstack{\\ matrix \\ struct} & vector & \shortstack{\\ Matrix dimensions\\ Data structure missing fields} \\ \hline
\end{tabular}
\end{center}

\subsection{Interface Semantics}

%\subsubsection{State Variables}
%None

\subsubsection{Assumption}
The Control Module and the Event Module handled the different cases correctly,
 ensuring that each energy function necessarily receives valid numerical input 
  for the case it is meant to handle.

%\subsubsection{Invariant}
%None

\subsubsection{Access Program Semantics}

\noindent \textbf{Input:}\\
The Energy Module functions require a matrix containing at least three columns of
numerical data, and a data structure containing the fields created by the 
Input Format Module. \\

\noindent \textbf{Exceptions:}\\
Potential exceptions may occur when given an inappropriately sized matrix or a 
data structure lacking the required fields.\\

\noindent \textbf{Output:}\\
The Energy Module will return vectors containing numerical solutions to
the differential equations it governs. \\


\subsubsection{Considerations}
Note: the three exported access functions are implemented in three separate function files. 

%%%%%%%%%%%%%%%%%%%%%%%%%%%%%%%%%%%%%%%%%%%%%%%%%%

\section{MIS of Event Modules}

\subsection{Module Name: event1, event2}

%\subsection{Uses}

%\subsubsection{Imported Constants}
%None

%\subsubsection{Imported Data Types}
%None

%\subsubsection{Imported Access Programs}
%None

\subsection{Interface Syntax}

%\subsubsection{Exported Constants}

%\subsubsection{Exported Data Types}
%None

\subsubsection{Exported Access Programs}
\begin{center}
\begin{tabular}{l l l l}
\hline
\textbf{Name} & \textbf{In} & \textbf{Out} & \textbf{Exceptions} \\ \hline
event1 & \shortstack{\\ real \\ vector \\ struct} & \shortstack{\\ real \\ int \\ int}
 & None \\ \hline
event2 & \shortstack{\\ real \\ vector \\ struct} & \shortstack{\\ real \\ int \\ int}
 & \shortstack{\\ Vector dimensions \\ Data structure missing fields} \\ \hline
\end{tabular}
\end{center}

\subsection{Interface Semantics}

%\subsubsection{State Variables}
%None

%\subsubsection{Assumption}
%None

%\subsubsection{Invariant}
%None

\subsubsection{Access Program Semantics}
\noindent \textbf{Input:}\\
The Event Module functions require a numerical value, a vector containing at 
least three numerical values, and a data structure containing the fields created by the 
Input Format Module. \\

\noindent \textbf{Exceptions:}\\
Potential exceptions may occur when given inappropriately sized
vectors or a data structure lacking the required fields.\\

\noindent \textbf{Output:}\\
The Event Module will return three numerical values.  \\

\subsubsection{Local Constants}
direction = 0\\
isterminal = 1\\

\subsubsection{Considerations}
Note: the two exported access functions are implemented in two separate function files.

%%%%%%%%%%%%%%%%%%%%%%%%%%%%%%%%%%%%%%%%%%%%%%%%%

\section{MIS of Plotting Module}

\subsection{Module Name: plots}

\subsection{Uses}

%\subsubsection{Imported Constants}
%None

%\subsubsection{Imported Data Types}
%None

%\subsubsection{Imported Access Programs}
%None

\subsection{Interface Syntax}

%\subsubsection{Exported Constants}
%None

%\subsubsection{Exported Data Types}
%None

\subsubsection{Exported Access Programs}
\begin{center}
\begin{tabular}{l l l l}
\hline
\textbf{Name} & \textbf{In} & \textbf{Out} & \textbf{Exceptions} \\ \hline
plots & \shortstack{\\ vector \\ matrix \\ vector \\ vector} & displays figures
 & \shortstack{\\ Matrix dimensions\\ Vector dimensions } \\ \hline
\end{tabular}
\end{center}

\subsection{Interface Semantics}

%\subsubsection{State Variables}
%None

\subsubsection{Assumption}
All vectors and matrices are handled correctly by the Control Module to be the correct size.

%\subsubsection{Invariant}
%None

\subsubsection{Access Program Semantics}
\noindent \textbf{Input:}\\
The Plotting Module takes a matrix containing at least two columns of numerical
data, and three different vectors containing numerical data.\\

\noindent \textbf{Exceptions:}\\
Potential exceptions may occur when given inappropriately sized
matrices and vectors.\\

\noindent \textbf{Output:}\\
The Plotting Module produces figures of the plotted data it has received as input.

\bmac{Is it okay that the Exceptions contradict the Assumption?}

%%%%%%%%%%%%%%%%%%%%%%%%%%%%%%%%%%%%%%%%%

\section{MIS of Output Verification Module}

\subsection{Module Name: verify\_output}

\subsubsection{Imported Data Types}
\textbf{Uses} Input Parameters Module \textbf{Imports} params := structure

\subsection{Interface Syntax}

\subsubsection{Exported Access Programs}

\begin{center}
	\begin{tabular}{l c c c}
		\hline
		\textbf{Name} & \textbf{In} & \textbf{Out} & \textbf{Exceptions} \\ \hline
		verify\_{output} & \shortstack{struct \\ vector \\ matrix \\ vector \\ vector} & none & \shortstack{\\ Matrix dimensions \\ Vector dimensions \\ Data structure missing fields}  \\
		\hline
	\end{tabular}
\end{center}

\subsection{Interface Semantics}

\subsubsection{Assumption}

All vectors and matrices are handled correctly by the Control Module to be the correct size.

\subsubsection{Access Program Semantics}

\noindent \textbf{Input:}\\
verify\_{output} will accept a data structure containing the fields created by the Input Format Module, three vectors containing numerical data, and a matrix containing at least two columns of numerical data.\\

\noindent \textbf{Exceptions:}\\
Potential exceptions may occur when given inappropriately sized matrices and vectors or a data structure lacking the required fields.\\  

\noindent \textbf{Output:}\\
None

%%%%%%%%%%%%%%%%%%%%%%%%%%%%%%%%%%%%%%%%%%%%%%%%%%%%

\section{MIS of Output Format Module}

\subsection{Module Name: output}

%\subsection{Uses}

%\subsubsection{Imported Constants}
%None

%\subsubsection{Imported Data Types}
%None

%\subsubsection{Imported Access Programs}
%None

\subsection{Interface Syntax}

%\subsubsection{Exported Constants}
%None

%\subsubsection{Exported Data Types}
%None

\subsubsection{Exported Access Programs}
\begin{center}
\begin{tabular}{l l l l}
\hline
\textbf{Name} & \textbf{In} & \textbf{Out} & \textbf{Exceptions} \\ \hline
output & \shortstack{\\ string (file) \\ vector \\ matrix \\ vector \\ vector
\\ vector \\ struct} & prints to file
 & \shortstack{\\ Matrix dimensions\\ Vector dimensions \\ Data structure missing fields} \\ \hline
\end{tabular}
\end{center}

\subsection{Interface Semantics}

%\subsubsection{State Variables}
%None

\subsubsection{Assumption}
All vectors and matrices are handled correctly by the Control Module to be the correct size.

%\subsubsection{Invariant}
%None

\subsubsection{Access Program Semantics}
\noindent \textbf{Input:}\\
The Output Format Module takes a string, a matrix containing at least two columns of numerical
data, four different vectors containing numerical data, and a data structure 
as produced by the Input Format Module.\\

\noindent \textbf{Exceptions:}\\
Potential exceptions may occur when given inappropriately sized
matrices and vectors or a data structure lacking the required fields.\\

\noindent \textbf{Output:}\\
The Output Format Module produces a file with the given string as a name, which contains a formatted list of data given by the input arrays and data structure.

%%%%%%%%%%%%%%%%%%%%%%%%%%%%%%%%%%%%%%%%%%%%%%%%%%%%%%%%%%%%%%%%

\section{Appendix}
\begin{table}[!h]
\caption{Standard Input Variables} \label{TblInputVar}
\renewcommand{\arraystretch}{1.2}
\noindent \begin{longtable*}{l l l c} 
  \toprule
  \textbf{Var} & \textbf{Typical Value}\\
  \midrule 
  $L$	& 1.5 \si[per-mode=symbol]	{\metre}
  \\
  $D$	& 0.412 \si[per-mode=symbol] {\metre}	
  \\
  $V_P$ & 0.05 \si[per-mode=symbol] {\cubic\metre}	
  \\
  $A_P$ & 1.2 \si[per-mode=symbol] {\square\metre}	
  \\
  $\rho_P$ & 1007 \si[per-mode=symbol] {\kilogram\per\cubic\metre}
  \\
  $T_\text{melt}^{P}$ &	44.2 \si[per-mode=symbol] {\celsius} 
  \\
  $C_P^S$ & 1760 \si[per-mode=symbol] {\joule\per\kilo\gram\per\celsius}
  \\
  $C_P^L$ & 2270 \si[per-mode=symbol] {\joule\per\kilo\gram\per\celsius} 
  \\
  $H_f$ & 211600 \si[per-mode=symbol] {\joule\per\kilo\gram} & 
  \\
  $A_C$ & 0.12 \si[per-mode=symbol] {\square\metre}
  \\
  $T_C$	& 50 \si[per-mode=symbol] {\celsius}
  \\
  $\rho_W$ & 1000 \si[per-mode=symbol] {\kilo\gram\per\cubic\metre} 
  \\
  $C_W$ & 4186 \si[per-mode=symbol] {\joule\per\kilo\gram\per\celsius}
  \\
  $h_C$ & 1000 \si[per-mode=symbol] {\watt\per\square\metre\per\celsius}
  \\
  $h_P$ & 1000 \si[per-mode=symbol] {\watt\per\square\metre\per\celsius} 
  \\
  $T_\text{init}$ & 40 \si[per-mode=symbol] {\celsius} 
  \\
  $t_\text{final}$ & 50000 \si[per-mode=symbol] {\second} 
  \\
  AbsTol & $10^{-10}$
  \\
  RelTol & $10^{-10}$
  \\
  \bottomrule
\end{longtable*}
\end{table}

\begin{table}[!h]
\caption{Possible Exceptions} \label{TblExceptions}
\renewcommand{\arraystretch}{1.2}
\noindent \begin{longtable*}{l l} 
  \toprule
  \textbf{Message ID} & \textbf{Error Message}\\
  \midrule
	input:L &error: Tank length must be $> 0$ \\ 
	input:diam &error: Tank diameter must be $> 0$ \\ 
	input:Vp &error: PCM volume must be $> 0$ \\ 
	input:VpVt &error: PCM volume must be $<$ tank volume \\ 
	input:Ap &error: PCM area must be $> 0$ \\
	input:rho\_p &error: rho\_p must be $> 0$ \\ 
	input:Tmelt &error: Tmelt must be $> 0$ and $< T_C$ \\ 
	input:C\_ps &error: C\_ps must be $> 0$ \\ 
	input:C\_pl &error: C\_pl must be $> 0$ \\ 
	input:Hf &error: Hf must be $> 0$ \\ 
	input:Ac &error: Ac must be $> 0$ \\ 
	input:Tc &error: Tc must be $> 0 \text{ and } < 100$ \\ 
	input:rho\_w &error: rho\_w must be $> 0$ \\ 
	input:C\_w &error: C\_w must be $> 0$ \\ 
	input:hc &error: hc must be $> 0$ \\ 
	input:hp &error: hp must be $> 0$ \\ 
	input:Tinit &error: Tinit must be $> 0$ and $< 100$ \\ 
	input:TcTinit &error: Tc must be $>$ Tinit \\ 
	input:TinitTmelt &error: Tinit must be $<$ Tmelt \\ 
	input:tfinal &error: tfinal must be $> 0$ \\ 
  \bottomrule
\end{longtable*}
\end{table}

\end{document}
