\documentclass[12pt]{article}

\usepackage{bm}
\usepackage{amsmath, mathtools}
\usepackage{amsfonts}
\usepackage{amssymb}
\usepackage{graphicx}
\usepackage{colortbl}
\usepackage{xr}
\usepackage{hyperref}
\usepackage{longtable}
\usepackage{xfrac}
\usepackage{tabularx}
\usepackage{float}
\usepackage{siunitx}
\usepackage{booktabs}
\usepackage{multirow}
\usepackage[section]{placeins}
\usepackage{caption}
\usepackage{fullpage}

\hypersetup{
	bookmarks=true,         % show bookmarks bar?
	colorlinks=true,       % false: boxed links; true: colored links
	linkcolor=red,          % color of internal links (change box color with linkbordercolor)
	citecolor=green,        % color of links to bibliography
	filecolor=magenta,      % color of file links
	urlcolor=cyan           % color of external links
}

%% Comments
\newif\ifcomments\commentstrue

\ifcomments
\newcommand{\authornote}[3]{\textcolor{#1}{[#3 ---#2]}}
\newcommand{\todo}[1]{\textcolor{red}{[TODO: #1]}}
\else
\newcommand{\authornote}[3]{}
\newcommand{\todo}[1]{}
\fi

\newcommand{\wss}[1]{\authornote{blue}{SS}{#1}}
\newcommand{\bmac}[1]{\authornote{red}{BM}{#1}}

\newcommand{\progname}{SWHS}

\begin{document}
\title{Module Interface Specification for Solar Water Heating Systems Incorporating Phase Change Material}
\author{Brooks MacLachlan}
\date{\today}

\maketitle

\tableofcontents

\section{Introduction}

The following document details the Module Interface Specifications for the implemented 
modules in a program simulation Solar Water Heating System with Phase Change Material.
It is intended to ease navigation through the program for design and maintenance purposes.\\
Complementary documents include the System Requirement Specifications and Module Guide. 

\section{Notation}

The following table summarizes the primitive data types used by \progname. \progname also uses some derived data types: arrays ,strings, and structures. Arrays are lists filled with elements of the same data type. Strings are arrays of characters. Structures contain pairs of keys and values, where keys are unique variable names used to identify their corresponding value, and values can be any data type. \progname also uses functions, which are defined by the data types of their inputs and outputs

\begin{center}
\renewcommand{\arraystretch}{1.2}
\noindent 
\begin{tabular}{c c c} 
\toprule 
\textbf{Data Type} & \textbf{Notation} & \textbf{Description}\\ 
\midrule
character & char & a single symbol or digit\\
integer & $\mathbb{Z}$ & a number without a fractional component in (-$\infty$, $\infty$) \\
real & $\mathbb{R}$ & any number in (-$\infty$, $\infty$)\\
\bottomrule
\end{tabular} 
\end{center}

\section{Module Decomposition}
The following table is taken directly from the Module Guide document for this project.
\begin{table}[!h]
	\centering
	\begin{tabular}{p{0.3\textwidth} p{0.6\textwidth}}
		\toprule
		\textbf{Level 1} & \textbf{Level 2}\\
		\midrule
		
		{Hardware-Hiding Module} & ~ \\
		\midrule
		
		\multirow{7}{0.3\textwidth}{Behaviour-Hiding Module} & Input Format Module\\
		& Input Parameters Module\\
		& Input Verification Module\\
		& Output Format Module\\
		& Output Verification Module\\
		& Temperature ODEs Module\\
		& Energy Equations Module\\ 
		& Control Module\\
		\midrule
		
		\multirow{3}{0.3\textwidth}{Software Decision Module} & {Sequence Data Structure Module}\\
		& ODE Solver Module\\
		& Plotting Module\\
		\bottomrule
		
	\end{tabular}
	\caption{Module Hierarchy}
	\label{TblMH}
\end{table}

\section{MIS of Control Module} \label{Main}
\subsection{Module}
main
\subsection{Uses}
parameters (\ref{Parameters}), load\_params (\ref{Load}), verify\_params (\ref{VerifyInput}), temperature (\ref{Temperature}), ODE Solvers module (\ref{ODE}), energy (\ref{Energy}), verify\_output (\ref{VerifyOutput}), plot (\ref{Plot}), output (\ref{Output})
\subsection{Syntax}
\subsubsection{Exported Access Programs}
\begin{center}
\begin{tabular}{c c c c}
\hline
\textbf{Name} & \textbf{In} & \textbf{Out} & \textbf{Exceptions} \\
\hline
main & string & - & - \\
\hline
\end{tabular}
\end{center}
\subsection{Semantics}
\subsubsection{State Variables}
$time$: array of reals \\
$tempW$: array of reals \\
$tempP$: array of reals \\
$latHeat$: array of reals \\
$eW$: array of reals \\
$eP$: array of reals \\ 
$eTot$: array of reals
\subsubsection{Environment Variables}
$win$: 2D array of pixels displayed on the screen
\subsubsection{Access Routine Semantics}
\begin{tabular}{l l p{12cm}}
main($s$): & transition: & $time$, $tempW$, $tempP$, $latHeat$ $eW$, $eP$, $eTot$, $win :=$ $results$[0], $results$[1], $results$[2], $results$[3], $eW1 \| eW2 \| eW3$, $eP1 \| eP2 \| eP3$, $post$($eW$[$i$]) $+$ $post$($eP$[$i$]) where $i \in$ [0..$|post$($eW$)$| - 1$], Prints information about the melting of PCM. \\
& exception: & none \\
\end{tabular}

\section{MIS of Input Parameters Module} \label{Parameters}
\subsection{Module}
parameters
\subsection{Uses}
N/A
\subsection{Syntax}
\subsubsection{Exported Data Types}
parameters := structure
\subsubsection{Exported Access Programs}
N/A
\subsection{Semantics}
\subsubsection{State Variables}
$params$.$L$: real \\
$params$.$diam$: real \\
$params$.$Vp$: real \\
$params$.$Ap$: real \\
$params$.$rho\_p$: real \\
$params$.$Tmelt$: real \\
$params$.$C\_ps$: real \\
$params$.$C\_pl$: real \\
$params$.$Hf$: real \\
$params$.$Ac$: real \\
$params$.$Tc$: real \\
$params$.$rho\_w$: real \\
$params$.$C\_w$: real \\
$params$.$hc$: real \\
$params$.$hp$: real \\
$params$.$Tinit$: real \\
$params$.$tstep$: real \\
$params$.$tfinal$: real \\
$params$.$AbsTol$: real \\
$params$.$RelTol$: real \\
$params$.$ConsTol$: real \\
$params$.$Vt$: real \\
$params$.$Mw$: real \\
$params$.$tau\_w$: real \\
$params$.$eta$: real \\
$params$.$Mp$: real \\
$params$.$tau\_ps$: real \\
$params$.$tau\_pl$: real \\
$params$.$Epmelt\_init$: real \\
$params$.$Ep\_melt3$: real \\ 
$params$.$Mw\_noPCM$: real \\
$params$.$tau\_w\_no\_PCM$: real
\subsubsection{Access Routine Semantics}
N/A

\section{MIS of Input Format Module} \label{Load}
\subsection{Module}
load\_params
\subsection{Uses}
parameters (\ref{Parameters})
\subsection{Syntax}
\subsection{Exported Access Programs}
\begin{center}
\begin{tabular}{c c c c}
\hline
\textbf{Name} & \textbf{In} & \textbf{Out} & \textbf{Exceptions} \\
\hline
load\_params & string & parameters &  - \\
\hline
\end{tabular}
\end{center}
\subsection{Semantics}
\subsubsection{State Variables}
$params$: parameters
\subsubsection{Assumptions}
The input string corresponds to an existing filename in the current directory. The input file is formatted correctly.
\subsubsection{Access Routine Semantics}
\begin{center}
\begin{tabular}{l l p{12cm}}
load\_params($s$): & transition: & $params$.$L$, $params$.$diam$, $params$.$Vp$, $params$.$Ap$, $params$.$rho\_p$, $params$.$Tmelt$, $params$.$C\_ps$, $params$.$C\_pl$, $params$.$Hf$, $params$.$Ac$, $params$.$Tc$, $params$.$rho\_w$, $params$.$C\_w$, $params$.$hc$, $params$.$hp$, $params$.$Tinit$, $params$.$tstep$, $params$.$tfinal$, $params$.$AbsTol$, $params$.$RelTol$, $params$.$ConsTol$, $params$.$Vt$, $params$.$Mw$, $params$.$tau\_w$, $params$.$eta$, $params$.$Mp$, $params$.$tau\_ps$, $params$.$tau\_pl$, $params$.$Epmelt\_init$, $params$.$Ep\_melt3$ $:=$ $p$[0], $p$[1], $p$[2], $p$[3], $p$[4], $p$[5], $p$[6], $p$[7], $p$[8], $p$[9], $p$[10], $p$[11], $p$[12], $p$[13], $p$[14], $p$[15], $p$[16], $p$[17], $p$[18], $p$[19], $p$[20], $p$[21], $p$[22], $p$[23], $p$[24], $p$[25], $p$[26], $p$[27], $p$[28], $p$[29], $p$[30], $p$[31], where $p$ is the array of parameters obtained from the input file $s$ \\
& exception: & none \\
\end{tabular}
\end{center}

\section{MIS of Input Verification Module} \label{VerifyInput}
\subsection{Module}
verify\_params
\subsection{Uses}
parameters (\ref{Parameters})
\subsection{Syntax}
\subsubsection{Exported Access Programs}
\begin{center}
\begin{tabular}{p{4cm} p{2cm} p{2cm} p{6cm}}
\hline
\textbf{Name} & \textbf{In} & \textbf{Out} & \textbf{Exceptions} \\
\hline
verify\_valid & parameters & - & badLength, badDiam, badPCMVolume, badPCMAndTankVol, badPCMArea, badPCMDensity, badMeltTemp, badCoilAndInitTemp, badCoilTemp, badPCMHeatCapSolid, badPCMHeatCapLiquid, badHeatFusion, badCoilArea, badWaterDensity, badWaterHeatCap, badCoilCoeff, badPCMCoeff, badInitTemp, badFinalTime, badInitAndMeltTemp \\
\hline
verify\_recommended & parameters & - & - \\
\hline
\end{tabular}
\end{center}
\subsection{Semantics}
\subsubsection{Environment Variables}
$win$: 2D array of pixels displayed on the screen.
\subsubsection{Assumptions}
The load\_params function has been called on $params$, so the variables have all been assigned a value.
\subsubsection{Access Routine Semantics}
\begin{center}
\begin{tabular}{l l p{8cm}}
verify\_valid($params$): & transition: & $win$: (error is thrown $\Rightarrow$ Prints error message) \\
& exceptions: & $exc :=$ ($params.L \leq 0 \Rightarrow$ badLength $|$ $params.diam \leq 0 \Rightarrow$ badDiam $|$ $params.Vp \leq 0 \Rightarrow$ badPCMVolume $|$ $params.Vp \geq params.Vt \Rightarrow$ badPCMAndTankVol $|$ $params.Ap \leq 0 \Rightarrow$ badPCMArea $|$ $params.rho\_p \leq 0 \Rightarrow$ badPCMDensity $|$ $params.Tmelt \leq 0 \vee params.Tmelt \geq params.Tc \Rightarrow$ badMeltTemp $|$ $params.Tc \leq params.Tinit \Rightarrow$ badCoilAndInitTemp $|$ $params.Tc \geq 100 \vee params.Tc \leq 0 \Rightarrow$ badCoilTemp $|$ $params.C\_ps \leq 0 \Rightarrow$ badPCMHeatCapSolid $|$ $params.C\_pl \leq 0 \Rightarrow$ badPCMHeatCapLiquid $|$ $params.Hf \leq 0 \Rightarrow$ badHeatFusion $|$ $params.Ac \leq 0 \Rightarrow$ badCoilArea $|$ $params.rho\_w \leq 0 \Rightarrow$ badWaterDensity $|$ $params.C\_w \leq 0 \Rightarrow$ badWaterHeatCap $|$ $params.hc \leq 0 \Rightarrow$ badCoilCoeff $|$ $params.hp \leq 0 \Rightarrow$ badPCMCoeff $|$ $params.Tinit \leq 0 \vee params.Tinit \geq 100 \Rightarrow$ badInitTemp $|$ $params.tfinal \leq 0 \Rightarrow$ badFinalTime $|$ $params.Tinit \geq params.Tmelt \Rightarrow$ badInitAndMeltTemp)  See Appendix (\ref{Appendix}) for the complete list of exceptions and associated error messages. \\ \\
verify\_recommended($params$): & transition: & $win$: (Warning is thrown $\Rightarrow$ Prints warning message) \\
& exception: & none
\end{tabular}
\end{center}

\section{MIS of Temperature ODEs Module} \label{Temperature}
\subsection{Module}
temperature
\subsection{Uses}
parameters (\ref{Parameters})
\subsection{Syntax}
\subsubsection{Exported Access Programs}
\begin{center}
\begin{tabular}{p{3cm} p{5cm} p{4cm} p{2cm}}
\hline
\textbf{Name} & \textbf{In} & \textbf{Out} & \textbf{Exceptions} \\
\hline
temperature1 & array of reals, array of reals, array of reals, parameters & array of functions & - \\
\hline
temperature2 & array of reals, array of reals, array of reals, array of reals, parameters & array of functions & - \\
\hline
temperature3 & array of reals, array of reals, array of reals, parameters & array of functions & - \\
\hline
event1 & array of reals, array of reals, array of reals, parameters & function & - \\
\hline
event2 & array of reals, array of reals, array of reals, array of reals, parameters & function & - \\
\hline
\end{tabular}
\end{center}
\subsection{Semantics}
\subsubsection{State Variables}
$time$: array of reals \\
$Tw1$: array of reals \\
$Tw2$: array of reals \\
$Tw3$: array of reals \\
$Tp1$: array of reals \\
$Tp2$: array of reals \\
$Tp3$: array of reals \\
$Qp2$: array of reals \\
$params$: parameters
\subsubsection{Assumptions}
The load\_params function has been called on $params$, so the variables have all been assigned a value. The verify\_valid function has been called on $params$, so no exceptions occur due to physically impossible values. 
\subsubsection{Access Routine Semantics}
\begin{center}
\begin{tabular}{l l p{8cm}}
temperature1($t$, $Tw1$, $Tp1$, $params$): & output: & $out :=$ \{$dTw: real \times real \rightarrow real$, $dTp: real \times real \rightarrow real$\} \\
& exception: & none \\ \\
temperature2($t$, $Tw2$, $Tp2$, $Qp2$, $params$): & output: & $out :=$ \{$dTw: real \times real \rightarrow real$, $dTp: real \times real \rightarrow real$, $Qp: real \times real \rightarrow real$\} \\
& exception: & none \\ \\
temperature3($t$, $Tw3$, $Tp3$, $params$): & output: & $out :=$ \{$dTw: real \times real \rightarrow real$, $dTp: real \times real \rightarrow real$\} \\
& exception: & none \\ \\
event1($t$, $Tw1$, $Tp1$, $params$): & output: & $out :=$ $Ev: real \times real \rightarrow real$ \\ 
& exception: & none \\ \\
event2($t$, $Tw2$, $T2p$, $Qp2$, $params$): & output: & $out :=$ $Ev: real \times real \rightarrow real$ \\
& exception: & none \\
\end{tabular}
\end{center}

\section{MIS of ODE Solver Module} \label{ODE}
\subsection{Module}
ODE Solver Module
\subsection{Uses}
N/A
\subsection{Syntax}
\subsubsection{Exported Access Programs}
\begin{center}
\begin{tabular}{p{2cm} p{5cm} p{5cm} p{2cm}}
\hline
\textbf{Name} & \textbf{In} & \textbf{Out} & \textbf{Exceptions} \\
\hline
solve & function, array of reals, array of reals, function, real, real & array of reals ($NUM\_EQS$ of them) & - \\
\hline 
\end{tabular}
\end{center}
\subsection{Semantics}
\subsubsection{State Variables}
$results$: array of reals ($NUM\_EQS$ of them)
\subsubsection{Local Constants}
$NUM\_EQS$: integer
\subsubsection{Access Program Semantics}
\begin{center}
\begin{tabular}{l l p{10cm}}
solve($f$, $domain$, $ics$, $events$, $abstol$, $reltol$) & output: & $out := results$ \\
& exceptions: & none \\
\end{tabular}
\end{center}

\section{MIS of Energy Module} \label{Energy}
\subsection{Module}
energy
\subsection{Uses}
parameters (\ref{Parameters})
\subsection{Syntax}
\subsubsection{External Access Programs}
\begin{center}
\begin{tabular}{p{3cm} p{6cm} p{3cm} p{2cm}}
\hline
\textbf{Name} & \textbf{In} & \textbf{Out} & \textbf{Exceptions} \\
\hline
energy1Wat & array of reals, parameters & array of reals & - \\
\hline
energy1PCM & array of reals, parameters & array of reals & - \\
\hline
energy2Wat & array of reals, parameters & array of reals & - \\
\hline
energy2PCM & array of reals, parameters & array of reals & - \\
\hline
energy3Wat & array of reals, parameters & array of reals & - \\
\hline
energy3PCM & array of reals, parameters & array of reals & - \\
\hline
\end{tabular}
\end{center}
\subsection{Semantics}
\subsubsection{State Variables}
$eW1$: array of reals \\
$eP1$: array of reals \\
$eW2$: array of reals \\
$eP2$: array of reals \\
$eW3$: array of reals \\
$eP3$: array of reals \\
$params$: parameters
\subsubsection{Assumptions}
The load\_params function has been called on $params$, so all variables have been assigned a value. The verify\_params function has been called on $params$, so there are no exceptions due to physically impossible values.
\subsubsection{Access Routine Semantics}
\begin{center}
\begin{tabular}{l l p{10cm}}
energy1Wat($Tw1$, $params$): & output: & $out := eW1[0..|Tw1|]$ \\ 
& exception: & none \\ \\
energy1PCM($Tp1$, $params$): & output: & $out := eP1[0..|Tp1|]$ \\ 
& exception: & none \\ \\
energy2Wat($Tw2$, $params$): & output: & $out := eW2[0..|Tw2|]$ \\ 
& exception: & none \\ \\
energy2PCM($Qp2$, $params$): & output: & $out := eP2[0..|Qp2|]$ \\ 
& exception: & none \\ \\
energy3Wat($Tw3$, $params$): & output: & $out := eW3[0..|Tw3|]$ \\ 
& exception: & none \\ \\
energy3PCM($Tp3$, $params$): & output: & $out := eP3[0..|Tp3|]$ \\
& exception: & none \\
\end{tabular}
\end{center}

\section{MIS of Output Verification Module} \label{VerifyOutput}
\subsection{Module}
verify\_output
\subsection{Uses}
parameters (\ref{Parameters})
\subsection{Syntax}
\subsubsection{Exported Access Programs}
\begin{center}
\begin{tabular}{p{3cm} p{7cm} p{2cm} p{2cm}}
\hline
\textbf{Name} & \textbf{In} & \textbf{Out} & \textbf{Exceptions} \\
\hline
verify\_output & array of reals, array of reals, array of reals, array of reals, array of reals, parameters & - & - \\
\hline
\end{tabular}
\end{center}
\subsection{Semantics}
\subsubsection{State Variables}
$time$: array of reals \\
$tempW$: array of reals \\
$tempP$: array of reals \\
$eW$: array of reals \\
$eP$: array of reals \\
$params$: parameters
\subsubsection{Environment Variables}
$win$: 2D array of pixels displayed on the screen
\subsubsection{Local Variables}
$errorWater$: real \\
$errorPCM$: real
\subsubsection{Assumptions}
The load\_params function has been called on $params$, so every variable in the structure has a value. The verify\_valid function has been called on $params$, so there are no exceptions due to physically impossible values. The temperature and energy arrays have been filled by the ODE solver and energy functions, so there is no divide by zero exception.
\subsubsection{Access Routine Semantics}
\begin{center}
\begin{tabular}{l l p{6cm}}
verify\_output($t$, $Tw$, $Tp$, $Ew$, $Ep$, $params$): & transition: & $win$: ($errorWater > ConsTol \vee errorPCM > ConsTol \Rightarrow$ Prints warning message(s)) \\
& exception: & none \\
\end{tabular}
\end{center}

\section{MIS of Plotting Module} \label{Plot}
\subsection{Module}
plot
\subsection{Uses}
N/A
\subsection{Syntax}
\subsubsection{Exported Access Programs}
\begin{center}
\begin{tabular}{p{2cm} p{8cm} p{2cm} p{2cm}}
\hline
\textbf{Name} & \textbf{In} & \textbf{Out} & \textbf{Exceptions} \\
\hline
plot & array of reals, array of reals, array of reals, array of reals, array of reals, string & - & - \\
\hline
\end{tabular}
\end{center}
\subsection{Semantics}
\subsubsection{State Variables}
$time$: array of reals \\
$tempW$: array of reals \\
$tempP$: array of reals \\
$eW$: array of reals \\
$eP$: array of reals \\
$filename$: string
\subsubsection{Environment Variables}
$directory$: The current directory of files from which the program is run.
\subsubsection{Access Routine Semantics}
\begin{center}
\begin{tabular}{l l p{6cm}}
plot($t$, $Tw$, $Tp$, $Ew$, $Ep$, $filename$): & transition: & $directory$: writes a .png file named $filename$ containing the graphs of the simulation results. \\
& exception: & none \\
\end{tabular}
\end{center}

\section{MIS of Output Module} \label{Output}
\subsection{Module}
output
\subsection{Uses}
parameters (\ref{Parameters})
\subsection{Syntax}
\subsubsection{Exported Access Program}
\begin{center}
\begin{tabular}{p{3cm} p{7cm} p{2cm} p{2cm}}
\hline
\textbf{Name} & \textbf{In} & \textbf{Out} & \textbf{Exceptions} \\
\hline
output & string, array of reals, array of reals, array of reals, array of reals, array of reals, array of reals, parameters & - & - \\
\hline
\end{tabular}
\end{center}
\subsection{Semantics}
\subsubsection{State Variables}
$params$: parameters \\
$time$: array of reals \\
$tempW$: array of reals \\
$tempP$: array of reals \\
$eW$: array of reals \\
$eP$: array of reals \\
$eTot$: array of reals \\
$filename$: string
\subsubsection{Environment Variables}
$directory$: The current directory of files from which the program is run.
\subsubsection{Assumptions}
The load\_params function was called on $params$, so all the variables have been assigned a value. The ODE solver and energy functions have filled $time$, $tempW$, $tempP$, $eW$, and $eP$ with results.
\subsubsection{Access Routine Semantics}
\begin{center}
\begin{tabular}{l l p{6cm}}
output($params$, $t$, $Tw$, $Tp$, $Ew$, $Ep$, $filename$): & transition: & $directory$: writes a .txt file named $filename$ containing the input parameters, calculated parameters, and results of the simulation. \\
& exception: & none \\
\end{tabular}
\end{center}

\section{Appendix} \label{Appendix}
\begin{center}
\begin{tabular}{l l}
\hline
\textbf{Message ID} & \textbf{Error Message} \\
\hline
badLength & Error: Tank length must be $> 0$ \\
badDiam & Error: Tank diameter must be $> 0$ \\
badPCMVolume & Error: PCM volume must be $> 0$ \\
badPCMAndTankVol & Error: PCM volume must be $<$ tank volume \\
badPCMArea & Error: PCM area must be $> 0$ \\
badPCMDensity & Error: rho\_p must be $> 0$ \\
badMeltTemp & Error: Tmelt must be $> 0$ and $< Tc$ \\
badCoilAndInitTemp & Error: Tc must be $>$ Tinit \\
badCoilTemp & Error: Tc must be $> 0$ and $< 100$ \\
badPCMHeatCapSolid & Error: C\_ps must be $> 0$ \\
badPCMHeatCapLiquid & Error: C\_pl must be $> 0$ \\
badHeatFusion & Error: Hf must be $> 0$ \\
badCoilArea & Error: Ac must be $> 0$ \\
badWaterDensity & Error: rho\_w must be $> 0$ \\
badWaterHeatCap & Error: C\_w must be $> 0$ \\
badCoilCoeff & Error: hc must be $> 0$ \\
badPCMCoeff & Error: hp must be $> 0$ \\
badInitTemp & Error: Tinit must be $> 0$ and $< 100$ \\
badFinalTime & Error: tfinal must be $> 0$ \\
badInitAndMeltTemp & Error: Tinit must be $<$ Tmelt \\
\hline
\end{tabular}
\end{center}
\end{document}