\documentclass[12pt]{article}

\usepackage{bm}
\usepackage{amsmath}
\usepackage{amsfonts}
\usepackage{amssymb}
\usepackage{graphicx}
\usepackage{colortbl}
\usepackage{xr}
\usepackage{hyperref}
\usepackage{longtable}
\usepackage{xfrac}
\usepackage{tabularx}
\usepackage{float}
\usepackage{siunitx}
\usepackage{booktabs}

%\usepackage{refcheck}

\hypersetup{
    bookmarks=true,         % show bookmarks bar?
      colorlinks=true,       % false: boxed links; true: colored links
    linkcolor=red,          % color of internal links (change box color with linkbordercolor)
    citecolor=green,        % color of links to bibliography
    filecolor=magenta,      % color of file links
    urlcolor=cyan           % color of external links
}
\newcommand{\NN}[1]{{\color{red}#1}}
\newcommand{\WSS}[1]{{\color{blue}#1}}

\newcommand{\colZwidth}{1.0\textwidth}
\newcommand{\blt}{- } %used for bullets in a list
\newcommand{\colAwidth}{0.13\textwidth}
\newcommand{\colBwidth}{0.82\textwidth}
\newcommand{\colCwidth}{0.1\textwidth}
\newcommand{\colDwidth}{0.05\textwidth}
\newcommand{\colEwidth}{0.8\textwidth}
\newcommand{\colFwidth}{0.17\textwidth}
\newcommand{\colGwidth}{0.5\textwidth}
\newcommand{\colHwidth}{0.28\textwidth}
\newcounter{defnum} %Definition Number
\newcommand{\dthedefnum}{GD\thedefnum}
\newcommand{\dref}[1]{GD\ref{#1}}
\newcounter{datadefnum} %Datadefinition Number
\newcommand{\ddthedatadefnum}{DD\thedatadefnum}
\newcommand{\ddref}[1]{DD\ref{#1}}
\newcounter{theorynum} %Theory Number
\newcommand{\tthetheorynum}{T\thetheorynum}
\newcommand{\tref}[1]{T\ref{#1}}
\newcounter{tablenum} %Table Number
\newcommand{\tbthetablenum}{T\thetablenum}
\newcommand{\tbref}[1]{TB\ref{#1}}
\newcounter{assumpnum} %Assumption Number
\newcommand{\atheassumpnum}{P\theassumpnum}
\newcommand{\aref}[1]{A\ref{#1}}
\newcounter{goalnum} %Goal Number
\newcommand{\gthegoalnum}{P\thegoalnum}
\newcommand{\gsref}[1]{GS\ref{#1}}
\newcounter{instnum} %Instance Number
\newcommand{\itheinstnum}{IM\theinstnum}
\newcommand{\iref}[1]{IM\ref{#1}}
\newcounter{reqnum} %Requirement Number
\newcommand{\rthereqnum}{P\thereqnum}
\newcommand{\rref}[1]{R\ref{#1}}
\newcounter{lcnum} %Likely change number
\newcommand{\lthelcnum}{LC\thelcnum}
\newcommand{\lcref}[1]{LC\ref{#1}}

\newcommand{\tclad}{T_\text{CL}}
\newcommand{\degree}{\ensuremath{^\circ}}
\newcommand{\progname}{SWHS}


\usepackage{fullpage}

\begin{document}

\title{Verification and Validation Plan for Solar Water Heating Systems Incorporating 
Phase Change Material} 
\author{Maya Grab}
\date{\today}
	
\maketitle

\tableofcontents

%%%%%%%%%%%%%%%%%%%%%%%%
%
%	1.) General Information 
%
%%%%%%%%%%%%%%%%%%%%%%%%

\section{General Informationl}
The following section provides an overview of the Verification and Validation (V\&V) Plan 
for a 
 This section explains the purpose of this document, the scope of the system,
  common definitions, acronyms and abbreviations that are used in the document,
   and an overview of the following sections

%1.1 Purpose
\subsection{Purpose}
The main purpose of this document is to describe the verification and validation 
process that will be used to test a simulation for 
This document is indented to be used as a reference for all future testing and will
be used to increase confidence in the software implementation.  

This document will be used as a starting point for the verification and validation report. The 
test cases presented within this document will be executed and the output will be analyzed to 
determine if the software is implemented correctly.  


%1.2 Scope
\subsection{Scope}


%1.3  Definitions, Acronyms, and abbreviations 
\subsection{Definitions, Acronyms, and Abbreviations }

\renewcommand{\arraystretch}{1.2}
\begin{tabular}{l l} 
  \toprule		
  \textbf{symbol} & \textbf{description}\\
  \midrule 
  QA		&Quality assurance\\
  SRS		&Software requirements specification\\
  V\&V		& Verification and validation\\
  V\&VP 	& Verification and validation plan\\
  V\&VR 	& Verification and validation report\\

  \bottomrule
\end{tabular}\\

%1.4 Overview of Document
\subsection{Overview of Document }
The following sections provide more detail about the V\&V of a . Information about the testing process is provided, and the software specifications
that were discussed in the SRS document are stated.  The evaluation process that will be followed during 
testing is outlined, and test cases for both the system testing and unit testing are provided. 

%%%%%%%%%%%%%%%%%%%%%%%%
%
%	2.) Plan
%
%%%%%%%%%%%%%%%%%%%%%%%%

\section{Plan}
This section provides a description of the software that is being tested, the team that will
perform the testing, the milestones for the testing phase, and the budget allocated to the testing. 

%2.1 Software Description
\subsection{Software Description}


%2.2 Test Team
\subsection{Test Team} 
The team that will execute the test cases, write and review the V\&VR consist of:


%2.3 Milestones
\subsection{Milestones}

%2.3.1 Location
\subsubsection{Location}
The location where the testing will be performed is Hamilton Ontario. The institution that
will be performing the testing is McMaster University. 


%2.3.1 Dates and Deadlines
\subsubsection{Dates and Deadlines}


%2.4 Budget
\subsection{Budget}
The budget for the testing of this system is being funded by McMaster University and NSERC

%%%%%%%%%%%%%%%%%%%%%%%%
%
%	3.) Software Specification
%
%%%%%%%%%%%%%%%%%%%%%%%%

\section{Software Specification}
This section provides the functional requirements, the business tasks that the
software is expected to complete, and the nonfunctional requirements, the
qualities that the software is expected to exhibit.

%3.1 Functional Requirements
\subsection{Functional Requirements}


%3.2 Nonfunctional Requirements
\subsection{Nonfunctional Requirements}



%%%%%%%%%%%%%%%%%%%%%%%%
%
%	4.) Evaluation
%
%%%%%%%%%%%%%%%%%%%%%%%%

\section{Evaluation}
This section first presents the methods and constraints that are to be used during
the evaluation process. This is followed by how the data obtained by the testing will be 
evaluated, which includes: how the data will be recorded, how to move from one test
to the next, and how to determine if the test was successful. 

%4.1 Methods and Constraints
\subsection{ Methods and Constraints} 

%4.1.1 Methodology
\subsubsection{Methodology} 


% 4.1.2 Extent of Testing
\subsubsection{Extent of Testing}

% 4.1.3 Test Tools
\subsubsection{Test Tools}

% 4.1.4 Testing Constraints
\subsubsection{ Testing Constraints}


% 4.2  Data Evaluation
\subsection{ Data Evaluation}

% 4.2.1 Data Recording
\subsubsection{Data Recording}
After each test is run the results of the test should be recorded in the following format: 
~\newline
Test ID: 
~\newline
Input:
~\newline
Expected Output:
~\newline
Actual Output:
~\newline
Result: 

% 4.2.2 Test Progression
\subsubsection{Test Progression}


% 4.2.3 Testing Criteria
\subsubsection{ Testing Criteria}


% 4.2.4 Test Data Reduction
\subsubsection{ Testing Data Reduction}


%%%%%%%%%%%%%%%%%%%%%%%%
%
%	5.) System Test Description
%
%%%%%%%%%%%%%%%%%%%%%%%%

\section{System Test Description}


%5.x Test identifier 
\subsection{Test Identifier}

% 5.x.1 Means of Control
\subsubsection{Means of Control}

% 5.x.2 Input
\subsubsection{ Input}


% 5.x.3 Expected Output
\subsubsection{ Expected Output}


% 5.x.4 Test  Procedures
\subsubsection{Procedure}

% 5.x.5  Preperation
\subsubsection{Preperation}


%%%%%%%%%%%%%%%%%%%%%%%%
%
%	6.) Unit Test Description 
%
%%%%%%%%%%%%%%%%%%%%%%%%
\section{Module Name}


\subsection{Module Information}


\subsubsection{Module Inputs}


\subsubsection{Module Outputs}


\subsubsection{Related Modules}


\subsection{Test Data}


\subsubsection{Inputs}


\subsubsection{Expected Outputs}

\end{document}


